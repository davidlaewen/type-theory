\section{General information}

This lecture concerns the verification of proofs using software,
also referred to as \emph{formalisation}.

In \emph{interactive theorem proving}, the programmer/user must
specify their proof using a form of programming language.
There are two main types of proof assistants:
\begin{itemize}
  \item Those using \emph{dependent types} (e.g. Coq, Agda, Lean)
  \item Those using \emph{higher-order logic} (e.g. Isabelle, HOL4)
\end{itemize}

\subsection{Some terminology}
\begin{infobox}
  \textbf{Curry-Howard-de Bruijn correspondence:}
  The \emph{Curry-Howard-de Bruijn correspondence} states that proofs in natural
  deduction correspond to terms in the lambda calculus.
  That is, proofs can be (en)coded as lambda calculus terms.

  If there exists a $\lambda$ term with a given type, the proposition corresponding to
  that type holds true.
  In Coq, the term for a proof can be displayed with the $\texttt{Print.}$ command. \br
  %
  \textbf{Types:}
  Since \emph{types} are ``axiomatic'' in type theory, they are somewhat difficult
  to define. There several views/interpretations of types.
\end{infobox}

\section{Meta Theory and Confluence of Reduction}
There are many interesting and important properties of type systems,
which can proven in \emph{meta theory}.
Some examples of such properties are:
\begin{itemize}
    \item Subject reduction (preservation of typing):
            If $\Gamma \turnstile M : A$ and $M \to_\beta N$, then $\Gamma \turnstile N : A$.
    \item Confluence of $\beta$-reduction:
            If $M \eval_\beta P_1$ and $M \eval_\beta P_2$, then $\exists Q: (P_1 \eval_\beta Q \land P_2 \eval_\beta Q)$.
    \item Normalisation:
    \begin{itemize}
        \item Weak normalisation (WN): A term $M$ is \emph{weakly normalising} if $\exists P \in \NF: (M \eval_\beta P)$,
                where $\NF$ denotes the set of all terms in normal form.
        \item Strong normalisation (SN): A term $M$ is \emph{strongly normalising} if
                $\neg \exists (P_i)_{i \in \Nat}: M = P_0$ and $P_0 \to_\beta P_1 \to_\beta P_2 \to_\beta ...$,
                i.e. there is no infinite reduction starting from $M$.
    \end{itemize}
\end{itemize}

We first discuss the subject reduction as an example of a meta-theoretic proof.
Many properties in meta theory are proven by induction on the typing derivation or the
term grammar.

\begin{lemma*}
    \upshape If $ \Gamma \turnstile M : A$ and $M \to_\beta N$, then $\Gamma \turnstile N : A$.
\end{lemma*}
\begin{proof}
    By induction on $M$.
\end{proof}

\subsection{Parallel reduction}
Properties of parallel reduction ($\paraRed$):
\begin{itemize}
    \item $M \paraRed M$
    \item If $M \to_\beta P$, then $M \paraRed P$.
    \item If $M \paraRed P$, then $M \eval_\beta P$.
\end{itemize}

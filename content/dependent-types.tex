\section{Dependent types}

The underlying principle of dependent types is the \emph{generalisation} of function types
by allowing types to \emph{depend} on terms.
We go from function types of the form $A \to B$ to the form $\Pi x : A. B$
(alternatively denoted as $\forall x : A. B$).

We choose to discuss dependent types before polymorphic types since dependent types correspond to
predicate logic, the most commonly used logic.
The extension of STLC/$\STLC$ with dependent types is typically denoted with $\lambda P$.
This new type system $\lambda P$ is the second out of a total of eight type systems contained in
the so-called \emph{lambda cube} that will be discussed in the course.

\subsection{Predicate logic}
More precisely, $\lambda P$ corresponds to \emph{minimal} predicate logic, which consists only
of quantification $\forall/\exists$ and implication $\to$,
but we will instead consider full predicate logic in natural deduction style.

\subsubsection{Syntax}

\begin{figure}[htbp]
    \[
    \begin{array}{r@{\hspace{.5em}}cl@{\hspace{2em}}r}
        \multicolumn{4}{c}{x,y \in \text{Variables}}\\
        \multicolumn{4}{c}{f \in \text{Function names}}\\
        \multicolumn{4}{c}{p \in \text{Predicate symbols}}\\[2mm]
        %
        M & \Coloneqq & x & \emph{Variable} \\
        & | & f(M,...,M) & \emph{Function application}
    \end{array}
    \]
    \caption{Term syntax of predicate logic}
    \Description{Term syntax of predicate logic}
    \label{fig:predicate-logic-term-syntax}
\end{figure}

\begin{figure}[htbp]
    \[
    \begin{array}{r@{\hspace{.5em}}cl@{\hspace{2em}}r}
        A & \Coloneqq & p(M,...,M) & \emph{Predicate application} \\
        & | & A \to A & \emph{Implication} \\
        & | & \forall x. A & \emph{Universal quantification} \\
        & | & \exists x. A & \emph{Existential quantification} \\
        & | & \top & \emph{True} \\
        & | & \bot & \emph{False} \\
        & | & A \land A & \emph{Conjunction} \\
        & | & A \lor A & \emph{Disjunction}
    \end{array}
    \]
    \caption{Proposition syntax of predicate logic}
    \Description{The syntax of propositions in predicate logic}
    \label{fig:predicate-logic-prop-syntax}
\end{figure}

\Cref{fig:predicate-logic-term-syntax} shows the term syntax of predicate logic,
assuming a finite number of functions $f$ with fixed arity.

\Cref{fig:predicate-logic-prop-syntax} shows the proposition syntax, with the first four cases
corresponding to those present in minimal predicate logic.

Let us consider the proposition $\forall x. p(x) \to q(x)$.
Typically, the $\forall$ quantifier binds \emph{weakly}, meaning that it would range only over $p(x)$,
not the full implication.
We instead use the notation of Coq, in which the $\forall$ ranges over the entire following proposition
$p(x) \to q(x)$.

Thus we must include parentheses around the first two clauses of the proposition
\[ (\forall x. p(x) \to q(x)) \to (\forall x. p(x)) \to \forall x. q(x) ,\]
which we will use as our running example.


\subsubsection{Proof rules}
The proof rules of predicate logic extend the rules we have previously seen for propositional logic
in \cref{fig:natural-deduction-rules}, namely the rules $\Intro \True$, $\Elim \False$, $\Intro \land$, $\Elim \land$$l$, $\Elim \land$$r$, $\Intro \lor$$l$, $\Intro \lor$$r$, $\Elim \lor$, $\Intro[x] \smallTo$ and $\Elim \smallTo$.

\begin{figure}[htbp]
    \centering
    %
    \def\extraVskip{3pt}
    \def\labelSpacing{6pt}
    %
    % Intro Forall
    \AxiomC{$A$}
    \RightLabel{$\Intro \forall$}
    \UnaryInfC{$\forall x. A$}
    \DisplayProof
    \hskip 12mm
    %
    % Elim Forall
    \AxiomC{$\forall x. A$}
    \RightLabel{$\Elim \forall$}
    \UnaryInfC{$A[x \coloneqq M]$}
    \DisplayProof\brbr
    %
    % Intro Exists
    \AxiomC{$A[x \coloneqq M]$}
    \RightLabel{$\Intro \exists$}
    \UnaryInfC{$\exists x. A$}
    \DisplayProof
    \hskip 12mm
    %
    % Elim Exists
    \AxiomC{$\exists x. A$}
    \AxiomC{$\forall x. (A \to C)$}
    \RightLabel{$\Elim \exists$}
    \BinaryInfC{$C$}
    \DisplayProof

    \caption{Proof rules of predicate logic}
    \Description{Proof rules of predicate logic}
    \label{fig:predicate-logic-rules}
\end{figure}

The additional rules in predicate logic are $\Intro \forall$, $\Elim \forall$,
$\Intro \exists$ and $\Elim \exists$, shown in \cref{fig:predicate-logic-rules}.
In rule $\Intro \forall$, the so-called \emph{variable condition} is imposed on $x$:
The variable $x$ must not occur in any \emph{open assumptions}, i.e. any assumptions
that have not yet been cancelled.
(Further, $x$ must be a free variable, not bound by any surrounding quantifier.)
Similarly, in $\Elim \exists$ the $x$ must not occur freely in $C$.

We can observe what happens otherwise using the following example proposition:
\[ \forall x. (p(x) \to \forall x. p(x)). \]
This proposition is equivalent to $\forall x. (p(x) \to \forall y. p(y))$, but we will purposefully
use a shadowed variable name instead.

\begin{center}
    \def\extraVskip{3pt}
    \def\labelSpacing{6pt}

    \AxiomC{\cancel{$p(x)$}}
    \RightLabel{$\Intro \forall$}
    \UnaryInfC{$\forall x. p(x)$}
    \RightLabel{$\Intro[H]\smallTo$}
    \UnaryInfC{$p(x) \to \forall x. p(x)$}
    \RightLabel{$\Intro \forall$}
    \UnaryInfC{$\forall x. (p(x) \to \forall x. p(x))$}
    \DisplayProof
    \brbr
\end{center}

Here, the second/upper application of $\Intro \forall$ does not satisfy the variable condition,
thus leading to an invalid proof if assumption $H$ were to be applied in the final step.

\subsubsection{Examples}
Let us now prove the following proposition using our new proof rules:
\[ (\exists x. \True) \to (\forall x.p(x)) \to \exists x. p(x) \]
We require the first premise $\exists x. \True$ in order to ensure that the domain of the
quantification over $x$ is inhabited. \\

\begin{center}
    \def\extraVskip{3pt}
    \def\labelSpacing{6pt}

    \AxiomC{$[(\exists x. \True)^{H_1}]$}


    \AxiomC{$[( \forall x. p(x) )^{H_3}]$}
    \RightLabel{$\Elim \forall$}
    \UnaryInfC{$p(y)$}
    \RightLabel{$\Intro \exists$}
    \UnaryInfC{$\exists x. p(x)$}
    %\AxiomC{$\forall x. ()$}
    \RightLabel{$\Intro[H_3]\smallTo$}
    \UnaryInfC{$(\forall x. p(x)) \to \exists x. p(x)$}
    \RightLabel{$\Intro[H_2]\smallTo$}
    \UnaryInfC{$\True \to (\forall x. p(x)) \to \exists x. p(x)$}
    \RightLabel{$\Intro \forall$}
    \UnaryInfC{$\forall y. (\True \to (\forall x. p(x)) \to \exists x. p(x))$}
    \RightLabel{$\Elim \forall$}
    \BinaryInfC{$(\forall x. p(x)) \to \exists x. p(x)$}
    \RightLabel{$\Intro[H_1]\smallTo$}
    \UnaryInfC{$(\exists x. \True) \to (\forall x. p(x)) \to \exists x. p(x)$}
    \DisplayProof
    \\[6mm]
\end{center}

Let us also prove our example proposition from earlier: \\

\begin{center}
    \def\extraVskip{3pt}
    \def\labelSpacing{6pt}


    \AxiomC{$[\forall x. p(x) \to q(x)]^{H_1}$}
    \RightLabel{$\Elim \forall$}
    \UnaryInfC{$p(x) \to q(x)$}

    \AxiomC{$[\forall x. p(x)]^{H_2}$}
    \RightLabel{$\Elim \forall$}
    \UnaryInfC{$p(x)$}

    \RightLabel{$\Elim\smallTo$}
    \BinaryInfC{$q(x)$}
    \RightLabel{$\Intro \forall$}
    \UnaryInfC{$\forall x. q(x)$}
    \RightLabel{$\Intro[H_2]\smallTo$}
    \UnaryInfC{$(\forall x. p(x)) \to \forall x. q(x)$}
    \RightLabel{$\Intro[H_1]\smallTo$}
    \UnaryInfC{$(\forall x. p(x) \to q(x)) \to (\forall x. p(x)) \to \forall x. q(x)$}
    \DisplayProof
    \brbr
\end{center}

In minimal predicate logic, the rules $\Intro \smallTo$ and $\Intro \forall$ correspond to
$\lambda$-abstraction in the proof term, while $\Elim \smallTo$ and $\Elim \forall$ correspond
to function application.
The proof term for the above proposition thus has the following form:
\[ \lambda H_1: ...\ .\lambda H_2: ...\ .\lambda x: ...\ . (H_1 x) (H_2 x) \]

\subsubsection{Curry-Howard-deBruijn correspondence}
Our four new proof rules $\Intro \forall$, $\Elim \forall$, $\Intro \exists$ and $\Elim \exists$
correspond to the Coq tactics \verb|intro|, \verb|apply|, $\verb|exists|\ M$ and \verb|elim|, respectively.

Further, we can extend our correspondence notion between proofs of propositions and programs:
\begin{itemize}
    \item A proof of $A \to B$ corresponds to a function from proofs of $A$ to proofs of $B$.
    \item A proof of $A \land B$ corresponds to a \emph{pair} of a proof $A$ and a proof of $B$.
    \item A proof of $\forall x. A$ corresponds to a \emph{dependent} function mapping any $x$
          to a proof of $A$ that \emph{depends} on $x$.
    \item A proof of $\exists x. A$ corresponds to a \emph{dependent} pair of an object $x$ and
          a proof of $A$, which again \emph{depends} on $x$.
\end{itemize}

In fact, we can express a function type $A \to B$ as a dependent type $\Pi x : A. B$ where
$x$ does not occur in $B$.


\subsection{Type system of $\lambda P$}
In simple type theory, types are either base types (e.g. \verb|nat|, \verb|bool|, ...) or they are
function types $A \to B$.
With the addition of dependent types, types may be parameterised with arbitrary other types and
can accept types as arguments.
For instance, the canonical example of dependent types is the type of \emph{vectors},
i.e. the list type parameterised with the length of the list.

But first, let us consider the definition of the list type (with elements of type \verb|nat|):

\[
    \begin{array}{l@{\hspace{1em}}cl@{\hspace{2em}}r}
        \verb|list| & :: & * & \emph{List type} \\
        \verb|nil| & :: & \verb|list| & \emph{Empty list} \\
        \verb|cons| & :: & \verb|nat| \to \verb|list| \to \verb|list| & \emph{Non-empty list}
    \end{array}\vspace{2mm}
\]

Here, the type \verb|list| also has a type and is typed with $*$.
Before we discuss the reason for this, let us now define the type of lists of natural numbers
with length $n$, the dependent vector type.

\[
    \begin{array}{l@{\hspace{1em}}cl@{\hspace{2em}}r}
        \verb|vec| & :: & \verb|nat| \to * & \emph{Vector type} \\
        \verb|vnil| & :: & \verb|vec|\,\, 0 & \emph{Empty vector} \\
        \verb|vcons| & :: & \Pi n : \verb|nat|.\:\: \verb|nat| \to \verb|vec|\,\, n \to \verb|vec|\,\, (S\ n)
        & \emph{Non-empty vector}
    \end{array}\vspace{2mm}
\]

Given this definition, the vector $(1\ \ 2\ \ 3)$ is written as
$\verb|vcons|\ \ 2\ \ 1\ \ (\verb|vcons|\ \ 1\ \ 2\ \ (\verb|vcons|\ \ 0\ \ 3))$.

Proofs of $\forall x. A$ are elements of type $\Pi x:D.\ A$, where $D$ is the \emph{domain} of quantification for $x$.


\subsubsection{Syntax}
For $\lambda\smallTo$, we defined the syntax of terms (in \cref{fig:lambda-calculus-grammar}) and types (in \cref{fig:stlc-types-grammar}) in two separate grammars.
In $\lambda P$ terms and types are merged and we no longer distinguish between the two.
Instead, everything is a term.

\begin{figure}[htbp]
    \[
    \begin{array}{r@{\hspace{.5em}}cl@{\hspace{2em}}r}
      \multicolumn{4}{c}{x,y \in \textsc{Var}}\\[2mm]
      t & \Coloneqq & x & \emph{Variables} \\
      & | & t\ t' & \emph{Applications} \\
      & | & \lambda x: t.\ t' & \emph{Abstractions} \\
      & | & \Pi x : t.\ t' & \emph{Function types} \\
      & | & s & \emph{Sorts} \br
      %
      s & \Coloneqq & \ast & \emph{Star} \\
      & | & \square & \emph{Box}
    \end{array}
    \]
    \caption{Syntax of $\lambda P$ preterms}
    \Description{Syntax of $\lambda P$ preterms}
    \label{fig:dependent-terms-grammar}
\end{figure}

The grammar of $\lambda P$ is shown in \cref{fig:dependent-terms-grammar}.
It describes the set of \emph{preterms} of $\lambda P$, which (as for $\lambda\smallTo$) includes terms that are not well-typed.
The term syntax includes a new notion of \emph{sorts}, encompassing the terms $\ast$ (\emph{star}) and $\square$ (\emph{box}).
The box type $\square$ serves to ``stratify'' the typing relation:
In $\lambda P$, every term other than $\square$ has a type. Repeatedly taking the type of any term will eventually
yield $\square$.
Reaching the sort level takes at most two steps from any term and since $\ast : \square$, one can take one more step
from $\ast$ to $\square$.
For instance, we get the following ``typing chain'' for the term $0$:
\[ 0 : \verb|nat|,\quad \verb|nat| : \ast,\quad \ast : \square. \]


\subsubsection{Typing rules}
We can define the type system of $\lambda P$ in two ways: Operationally, by specifying a type checking algorithm,
or inductively, be specifying the typing derivation rules.
An algorithm for typing a given term is given as follows:
\[
    \begin{array}{rcl}
        \type(\ast) & = & \square \\
        \type(A \to B) & = & \type(B) \\
        \type(\Pi x : A.\ B) & = & \type(B) \\
        \type(\lambda x : A.\ M) & = & \Pi x : A.\ \type(M) \\
        \type(M\ M') & = & \hdots
    \end{array}
\]

The type system of $\lambda P$ consists of the typing rules shown in \cref{fig:dependent-typing-rules}.

\begin{figure}[htbp]
    \centering
    %
    \def\extraVskip{3pt}
    \def\labelSpacing{3pt}
    %
    % Axiom/Start
    \AxiomC{}
    \RightLabel{\textsc{Start}}
    \UnaryInfC{$\turnstile \ast : \square$}
    \DisplayProof
    \hskip 12mm
    %
    % Function types
    \AxiomC{$\Gamma \turnstile A : \ast$}
    \AxiomC{$\Gamma, x : A \turnstile B : s$}
    \RightLabel{\textsc{Prod}}
    \BinaryInfC{$\Gamma \turnstile (\Pi x : A.\ B) : s$}
    \DisplayProof
    \brbr
    %
    % Abstraction
    \AxiomC{$\Gamma, x : A \turnstile M : B$}
    \AxiomC{$\Gamma \turnstile (\Pi x : A.\ B) : s$}
    \RightLabel{\textsc{Abs}}
    \BinaryInfC{$\Gamma \turnstile (\lambda x : A.\ M) \;:\; \Pi x : A.\ B$}
    \DisplayProof
    \brbr
    %
    % Application
    \AxiomC{$\Gamma \turnstile F \;:\; \Pi x : A.\ B$}
    \AxiomC{$\Gamma \turnstile M : A$}
    \RightLabel{\textsc{App}}
    \BinaryInfC{$\Gamma \turnstile F\ M : B[x \coloneqq M]$}
    \DisplayProof
    \brbr
    %
    % Variable
    \AxiomC{$\Gamma \turnstile A : s$}
    \RightLabel{\textsc{Var}}
    \UnaryInfC{$\Gamma, x : A \turnstile x : A$}
    \DisplayProof
    \hskip 12mm
    %
    % Variable v2
    \AxiomC{$\Gamma \turnstile M : A$}
    \AxiomC{$\Gamma \turnstile B : s$}
    \RightLabel{\textsc{Weaken}}
    \BinaryInfC{$\Gamma, y : B \turnstile M : A$}
    \DisplayProof
    \brbr
    %
    % β-equivalence
    \AxiomC{$\Gamma \turnstile M : A$}
    \AxiomC{$A =_\beta A'$}
    \AxiomC{$\Gamma \turnstile A' : s$}
    \RightLabel{\textsc{Convert}}
    \TrinaryInfC{$\Gamma \turnstile M : A'$}
    \DisplayProof

    \caption{Typing rules of $\lambda P$}
    \Description{Typing rules of $\lambda P$}
    \label{fig:dependent-typing-rules}
\end{figure}

In the premises of the rules \textsc{Abs}, \textsc{Var}, \textsc{Weaken} and \textsc{Convert},
the $s$ stands for either $\ast$ or $\square$.
That is, each of these four rules is actually a shorthand for two rules, one where $s$ is $\ast$ and
another where $s$ is $\square$.
The \textsc{Start} rule is the only rule without a premise.
Accordingly, all typing derivations in $\lambda P$ begin with the \textsc{Start} rule.

If we can derive a judgement $\Gamma \turnstile M : A$ for some environment $\Gamma$, then $M$ and $A$ are
\emph{terms}. Not all preterms defined in \cref{fig:dependent-terms-grammar} are terms.


\subsubsection{Example} To illustrate the typing rules in \cref{fig:dependent-typing-rules}, let us derive the
type of the identity function $\lambda x : a.\ x$.

\begin{prooftreecustom}

    \AxiomC{}
    \RightLabel{\textsc{Start}}
    \UnaryInfC{$\turnstile \ast : \square$}
    \RightLabel{\textsc{Var}}
    \UnaryInfC{$a : \ast \turnstile a : \ast$}
    \RightLabel{\textsc{Var}}
    \UnaryInfC{$a : \ast,\ x : a \turnstile x : a$}

    \AxiomC{}
    \RightLabel{\textsc{Start}}
    \UnaryInfC{$\turnstile \ast : \square$}
    \RightLabel{\textsc{Var}}
    \UnaryInfC{$a : \ast \turnstile a : \ast$}


    \AxiomC{}
    \RightLabel{\textsc{Start}}
    \UnaryInfC{$\turnstile \ast : \square$}
    \RightLabel{\textsc{Var}}
    \UnaryInfC{$a : \ast \turnstile a : \ast$}

    \AxiomC{}
    \RightLabel{\textsc{Start}}
    \UnaryInfC{$\turnstile \ast : \square$}
    \RightLabel{\textsc{Var}}
    \UnaryInfC{$a : \ast \turnstile a : \ast$}

    \RightLabel{\textsc{Weaken}}
    \BinaryInfC{$a : \ast,\ x : a \turnstile a : \ast$}
    \RightLabel{\textsc{Prod}}
    \BinaryInfC{$a : \ast \turnstile (\Pi x : a.\ a) : \ast$}

    \RightLabel{\textsc{Abs}}
    \BinaryInfC{$a : \ast \turnstile (\lambda x : a.\ x) : \Pi x : a.\ a$}
\end{prooftreecustom}

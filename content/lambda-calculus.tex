\section{Brief overview of the lambda calculus}

There are two main forms of the lambda calculus: The untyped lambda calculus
as well as many different flavours of typed lambda calculus.

The grammar of the untyped lambda calculus is defined as follows:

\begin{figure}[htbp]
  \[
  \begin{array}{r@{\hspace{.5em}}cl@{\hspace{2em}}r}
    \multicolumn{4}{c}{x,y \in \textsc{Var}}\\[2mm]
    t, t_i & \Coloneqq & x & \emph{Variable} \\
    & | & (t_1\ t_2) & \emph{Application} \\
    & | & (\lambda x. t) & \emph{Abstraction}
  \end{array}
  \]
  \caption{The syntax of lambda calculus terms}
  \Description{The syntax of lambda calculus terms}
  \label{fig:lambda-calculus-grammar}
\end{figure}

\subsection{Notational conventions}
Application is left-associative, e.g.:
\[ xyz \equiv ((x y) z) \]

Multiple abstractions can be contracted, e.g.:
\[ \lambda x. \lambda y. \cdots \;\equiv\; \lambda x y. \cdots \]


Abstractions bind weakly, e.g.:
\[ \lambda x. t_1\ t_2 \equiv \lambda x. (t_1\ t_2) \]

\subsection{Combinator terms}
The following terms are used in \emph{combinatory logic} and referred to as
\emph{combinators}:

\[
\begin{array}{ccl}
  \mathbf{I} &\coloneqq& \lambda x. x \\
  \mathbf{K} &\coloneqq& \lambda x y. x \\
  \mathbf{S} &\coloneqq& \lambda x y z. x z (y z)
\end{array}
\]


\subsection{Equivalence and reduction}
Two terms $t_1$ and $t_2$ are \emph{equivalent}, written as $t_1 \equiv t_2$, if they
denote the same AST, i.e. they only differ in parenthesis syntax.

Two terms $t_1$ and $t_2$ are $\alpha$-equivalent, written as $t_1 =_\alpha t_2$, if
they differ only in the naming of their variables, i.e. one term can be transformed
into the other through consistent renaming of bound variables.

The evaluation of lambda calculus terms is formalised in $\beta$-reduction, denoted
by $\to_\beta$.
The rule of $\beta$-reduction is written as
\[ (\lambda x. M) N \to_\beta M[x \coloneqq N], \]
where $M[x \coloneqq N]$ is equivalent to the term $M$ after replacing all bound
occurrences of $x$ with $N$.
Further, the reduction $M \to_\beta M'$ implies the following reductions:
\[
\begin{array}{ccc}
  \lambda x. M &\to_\beta& \lambda x. M' \\
  M N &\to_\beta& M' N \\
  N M &\to_\beta& N M'
\end{array}
\]
%
That is, $\beta$-reduction may also be applied within subterms.

We denote the reflective transitive closure of $\to_\beta$ with
$\twoheadrightarrow_\beta$ and define $\beta$-equivalence as an equivalence relation
$=_\beta$, whereby $t_1 =_\beta t_2$ \emph{iff} there exists a term $t'$ with
$t_1 \twoheadrightarrow_\beta t'$ and $t_2 \twoheadrightarrow t'$.

In the lambda calculus, $\beta$-reduction is \emph{confluent}, meaning that if the
reduction of a term diverges (when $\to_\beta$ can be applied in multiple positions),
the reduction will converge again and all reduction chains will lead to the same result.

\subsection{Termination}
A term is said to be in \emph{normal form} when it cannot be $\beta$-reduced any further.
This is the case when the term does not feature any function applications $M\ N$.
The reduction of a term is said to \emph{terminate} if applying $\to_\beta$ a finite
number of times yields a term in normal form.

In the general case, reduction does not terminate in the untyped lambda calculus.
For instance, applying $\beta$-reduction to the term
$\Omega \coloneqq (\lambda x. x\ x)\ (\lambda x. x\ x)$
yields the term $\Omega$ itself, i.e. the term never reaches normal form.

\subsection{Variable capture}
Consider the reduction of the term $\mathbf{K}MN \equiv (\lambda x y. x)\ M\ N$:
\begin{align*}
  & (\lambda x y. x)\ M\ N \\
  \to_\beta\ & (\lambda y. x)[x \coloneqq M] N \\
  \to_\beta\ & (\lambda y. M) N
\end{align*}
%
If the term $M$ happens to contain unbound occurrences of the variable $y$, these
occurrences would be \emph{captured} when $M$ is substituted into the body of
$\lambda y. \cdots$.
In order prevent this ``unexpected'' behaviour, one must $\alpha$-rename $y$ and all its
bound occurrences (there are none in this example) before performing the substitution
of $x$ with $M$.

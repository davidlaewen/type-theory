\section{Simply-typed lambda calculus}
The \emph{simply-typed lambda calculus} (\emph{STLC}), also referred to as
\emph{simple type theory} (\emph{STT}) and denoted by $\STLC$ is an extension of
the lambda calculus we have discussed so far with a \emph{type system}.
We use \emph{Church-style} notation, meaning that we annotate our terms and all binding variable
occurrences within them with their corresponding types, e.g.:
\[ \mathbf{K} \:\coloneqq\: \lambda x : A.\ \lambda y : B.\ x \:\ : \:\ A \to B \to A \]
%
Types and contexts for $\STLC$ are defined as follows:

\begin{figure}[htbp]
  \[
  \begin{array}{r@{\hspace{.5em}}cl@{\hspace{2em}}r}
    \multicolumn{4}{c}{x,y \in \textsc{Var} \quad \quad a,b \in \textsc{BaseType}}\\[2mm]
    A, B & \Coloneqq & a & \emph{Base types} \\
    & | & A \to B & \emph{Function types} \br
    \Gamma & \Coloneqq & \epsilon \quad | \quad \Gamma,\ x : A & \emph{Contexts}
  \end{array}
  \]
  \caption{Syntax of STLC types and contexts}
  \Description{Syntax of STLC types and contexts}
  \label{fig:stlc-types-grammar}
\end{figure}

The grammar remains nearly identical to that shown in \cref{fig:lambda-calculus-grammar}, except that
lambda abstractions $\lambda x.\ M$ now feature type annotations, i.e. $\lambda x : A.\ M$.
Typing judgments have the form $\Gamma \turnstile M : A$, stating that term $M$ has type $A$
in context $\Gamma$.

We now define the typing rules for $\STLC$:

\begin{figure}[htbp]
  \centering
  %
  \def\extraVskip{3pt}
  \def\labelSpacing{4pt}
  %
  % Var
  \AxiomC{$(x : A) \in \Gamma$}
  \RightLabel{\textsc{T-Var}}
  \UnaryInfC{$\Gamma \turnstile x : A$}
  \DisplayProof \hskip 4mm
  %
  % Application
  \AxiomC{$\Gamma \turnstile t_1 : A \to B$}
  \AxiomC{$\Gamma \turnstile t_2 : A$}
  \RightLabel{\textsc{T-App}}
  \BinaryInfC{$\Gamma \turnstile t_1\ t_2 : B$}
  \DisplayProof \hskip 4mm
  %
  % Abstraction
  \AxiomC{$\Gamma, x : A \turnstile t : B$}
  \RightLabel{\textsc{T-Abs}}
  \UnaryInfC{$\Gamma \turnstile \lambda x : A.\ t\ :\ A \to B$}
  \DisplayProof

  \caption{Typing rules of $\STLC$}
  \Description{Typing rules of $\STLC$}
  \label{fig:stlc-typing-rules}
\end{figure}

So far, we have referred to all expressions described by the grammar in
\cref{fig:lambda-calculus-grammar} as \emph{terms}. However, the term syntax permits the
construction of terms that are not \emph{well-typed}, i.e. for which there does not exist
some typing derivation $\Gamma \turnstile M : A$.
To differentiate more clearly, we will instead refer to all expressions described by the
term syntax as \emph{pre-terms} and only to well-typed pre-terms as \emph{terms}.

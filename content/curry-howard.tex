\section{Curry-Howard-de Bruijn correspondence}

The \emph{Curry-Howard-de Bruijn correspondence} denotes a correspondence between
logic and computation.
More specifically, proofs in logic correspond to terms in the lambda calculus
while logical propositions correspond to types.
Providing a proof for a proposition corresponds to providing a term
with a given type, showing that is inhabited.
Conversely, false propositions correspond to empty (or uninhabited) types.

For instance, the logical $\top$ (true) can be thought of as equivalent to the
\emph{Unit} type $\mathbb{1}$, while $\bot$ (false) corresponds to the empty type
$\mathbb{0}$.
Conjunction $A \land B$ corresponds to the pair/tuple type $A \times B$ (cartesian product), while
disjunction $A \lor B$ corresponds to the \emph{Either} type $A + B$ (disjunct sum).
Finally, implications $A \to B$ correspond to function types $A \to B$.

\subsection{Brouwer-Heyting-Kolmogorov (BHK) interpretation}
\emph{Propositional logic} includes negation $\neg$, conjunction $\land$, disjunction $\lor$, implication $\to$ and equivalence $\leftrightarrow$.
In the weaker subset of \emph{intuitionistic propositional logic}, the same connectives exist, but the following laws do not hold:
\begin{itemize}
    \item $A \lor \neg A$: Law of excluded middle
    \item $\neg \neg A \to A$: Double negation elimination
    \item $((A \to B) \to A) \to A$: Peirce's law
\end{itemize}
%
These three laws are equivalent: Each of the laws implies the other two (proofs in \texttt{hw03.v}).

An even weaker subset of intuitionistic logic is \emph{minimal propositional logic}, in which implication $(\to)$
is the only connective. Minimal propositional logic corresponds to simple type theory.

\subsection{Propositional logic}
We define the syntax of propositional logic as follows:
\begin{figure}[htbp]
    \[
    \begin{array}{r@{\hspace{.5em}}cl@{\hspace{.2em}}r}
        \multicolumn{4}{c}{a,b,c \in \textsc{Atom}}\\[2mm]
        A, B, C & \Coloneqq & a & \emph{Atomic propositions} \\
        & | & \top & \emph{True} \\
        & | & \bot & \emph{False} \\
        & | & (A \land B) & \emph{Conjunction} \\
        & | & (A \lor B) & \emph{Disjunction} \\
        & | & (A \to B) & \emph{Implication} \\
    \end{array}
    \]
    \caption{Syntax of propositional logic}
    \Description{Syntax of propositional logic}
    \label{fig:prop-logic-grammar}
\end{figure}

Negation $(\neg)$ appears to be missing here, but $\neg A$ is in fact just a shorthand for $A \to \bot$.
In addition, $A \leftrightarrow B$ is also just ``syntactic sugar'' for $(A \to B) \land (B \to A)$.

The rules of propositional logic are those of \emph{natural deduction}, shown in \cref{fig:natural-deduction-rules}.
They are split into \emph{introduction} and \emph{elimination} rules.

\begin{figure}[H]
    \centering
    %
    \def\extraVskip{3pt}
    \def\labelSpacing{6pt}
    % Intro True
    \AxiomC{}
    \RightLabel{$\Intro\top$}
    \UnaryInfC{$\top$}
    \DisplayProof
    \hskip 12mm
    %
    % Elim False
    \AxiomC{$\bot$}
    \RightLabel{\textsc{E$\bot$}}
    \UnaryInfC{$A$}
    \DisplayProof\brbr
    %
    % Intro And
    \AxiomC{$A$}
    \AxiomC{$B$}
    \RightLabel{$\Intro\land$}
    \BinaryInfC{$A \land B$}
    \DisplayProof \hskip 12mm
    %
    % Elim And Left
    \AxiomC{$A \land B$}
    \RightLabel{\text{E$\land$$l$}}
    \UnaryInfC{$A$}
    \DisplayProof \hskip 12mm
    %
    % Elim And Right
    \AxiomC{$A \land B$}
    \RightLabel{\text{E$\land$$r$}}
    \UnaryInfC{$B$}
    \DisplayProof\brbr
    %
    % Intro Or Left
    \AxiomC{$A$}
    \RightLabel{\text{I$\lor$$l$}}
    \UnaryInfC{$A \lor B$}
    \DisplayProof \hskip 12mm
    %
    % Intro Or Right
    \AxiomC{$B$}
    \RightLabel{\text{I$\lor$$r$}}
    \UnaryInfC{$A \lor B$}
    \DisplayProof \hskip 12mm
    %
    % Elim Or
    \AxiomC{$A \lor B$}
    \AxiomC{$A \to C$}
    \AxiomC{$B \to C$}
    \RightLabel{\text{E$\lor$}}
    \TrinaryInfC{$C$}
    \DisplayProof\brbr
    %
    % Intro Impl
    \AxiomC{$\begin{matrix}[A^x]\\[-1.5mm]\vdots\\ B\end{matrix}$}
    \RightLabel{$\Intro[x]\smallTo$}
    \UnaryInfC{$A \to B$}
    \DisplayProof \hskip 12mm
    %
    % Elim Impl
    \AxiomC{$A \to B$}
    \AxiomC{$A$}
    \RightLabel{$\Elim \smallTo$}
    \BinaryInfC{$B$}
    \DisplayProof

    \caption{The rules of natural deduction}
    \Description{The rules of natural deduction}
    \label{fig:natural-deduction-rules}
\end{figure}

In the rules, the context containing our assumptions is kept implicit, we simply use markers
$x, y, z, \hdots$ to indicate where we use which assumption.
Alternatively, we could also make the context $\Gamma$ explicit. Using this notation, the rules for
minimal logic are written as follows: \\

\begin{center}
    \def\extraVskip{3pt}
    \def\labelSpacing{6pt}
    %
    % Context
    \AxiomC{$A \in \Gamma$}
    \UnaryInfC{$\Gamma \turnstile A$}
    \DisplayProof \hskip 12mm
    %
    % Intro Impl
    \AxiomC{$\Gamma, A \turnstile B$}
    \RightLabel{$\Intro[x]\smallTo$}
    \UnaryInfC{$A \to B$}
    \DisplayProof \hskip 12mm
    %
    % Elim Impl
    \AxiomC{$\Gamma \turnstile A \to B$}
    \AxiomC{$\Gamma \turnstile A$}
    \RightLabel{$\Elim \smallTo$}
    \BinaryInfC{$\Gamma \turnstile B$}
    \DisplayProof
    \\[6mm]
\end{center}


\subsubsection{Example}
Let us prove $a \lor b \to b \lor a$. \\

\begin{center}
    \def\extraVskip{3pt}
    \def\labelSpacing{6pt}

    \AxiomC{$[(a \lor b)^x]$}

    \AxiomC{$[a^y]$}
    \RightLabel{I$\lor$$r$}
    \UnaryInfC{$b \lor a$}
    \RightLabel{$\Intro[y]\smallTo$}
    \UnaryInfC{$a \to b \lor a$}

    \AxiomC{$[b^z]$}
    \RightLabel{I$\lor$$l$}
    \UnaryInfC{$b \lor a$}
    \RightLabel{$\Intro[z]\smallTo$}
    \UnaryInfC{$b \to b \lor a$}

    \RightLabel{E$\lor$}
    \TrinaryInfC{$b \lor a$}
    \RightLabel{$\Intro[x]\smallTo$}
    \UnaryInfC{$a \lor b \to b \lor a$}
    \DisplayProof
\end{center}


\subsubsection{Coq tactics}
The derivation rules in \cref{fig:natural-deduction-rules} correspond to the following tactics in Coq:

\begin{center}
    \begin{tabular}{l|l}
        Coq tactic & Derivation rule(s) \\
        \hline
        \verb|intro| $x$. & $\Intro[x]\smallTo$ \\
        \verb|apply| $x$. & Zero or more $\Elim \smallTo$ and $[...^x]$ \\
        \verb|elim|. & $\Elim \land$, $\Elim \lor$ and $\Elim \bot$ \\
        \verb|split|. & $\Intro \land$ \\
        \verb|left|. & $\Intro\lor$$l$ \\
        \verb|right|. & $\Intro\lor$$r$
    \end{tabular}
\end{center}


\subsubsection{Detours}
In lambda calculus, a $\beta$-redex has the form $(\lambda x : A.\ M) N$.
By the Curry-Howard correspondence, a $\beta$-redex corresponds to a so-called \emph{detour} within a proof in minimal logic,
which consists of introducing an implication with $\Intro[\hdots]\smallTo$ and then immediately eliminating
the same implication with $\Elim \smallTo$.
In general, such a detour can be removed via \emph{detour elimination} according to the following rule:

\begin{center}
    \begin{minipage}{.4\textwidth}
        \begin{prooftreecustom}
            \AxiomC{$ \begin{matrix} [A^x]\\[-1.5mm] \vdots\\ B \end{matrix}$}
            \RightLabel{$\Intro[x]\smallTo$}
            \UnaryInfC{$A \to B$}

            \AxiomC{$\vdots$}\noLine
            \UnaryInfC{$A$}

            \RightLabel{$\Elim\smallTo$}
            \BinaryInfC{$B$}
        \end{prooftreecustom}
    \end{minipage}%
    \begin{minipage}{.1\textwidth}
        \[ \to \]
    \end{minipage}%
    \begin{minipage}{.2\textwidth}
        \[ \begin{matrix} \vdots\\ A\\ \vdots\\ B \end{matrix} \]
    \end{minipage}
\end{center}\vspace{2mm}

A proof containing no detours is said to be in \emph{normal form}.
As such, \emph{normalisation} of arbitrary proofs can be performed via repeated detour elimination,
which corresponds to normalisation of terms via $\eval_\beta$.

Similarly to $\beta$-reduction, proof normalisation does not necessarily shrink a proof tree:
In the rule above, $[A^x]$ may be used many times in the proof of $B$.
Replacing each occurence of $[A^x]$ with the full proof of $A$ may therefore exponentially increase
the size of the derivation tree.
